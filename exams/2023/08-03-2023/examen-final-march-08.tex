\documentclass[10pt,a4paper]{article}
\usepackage[paper=a4paper,hmargin=0.5cm,bottom=1.5cm]{geometry}
\usepackage[utf8]{inputenc}
\usepackage[spanish]{babel}
\usepackage{xargs}
\usepackage{xspace}
\usepackage{caratula}
\usepackage{ifthen}
\usepackage{aed2-tad,aed2-symb,aed2-itef,aed2-diseno}
\usepackage{aed2-tad,aed2-symb,aed2-itef}
\usepackage{algorithmicx, algpseudocode, algorithm}
\usepackage{graphicx}


\materia{Algoritmos y Estructuras de Datos II}
\titulo{Examen Final}
\subtitulo{8 de marzo 2023}
\integrante{Ezequiel Rueda Sanchez}{522/16}{ezequiel.ruedasanchez@gmail.com}


\begin{document}

\section{Ejercicio 1}

Determine si las siguientes afirmaciones son verdaderas o falsas. Justifique.
\newline
\newline
a) El invariante vale en la pre pero no en la post de todas las funciones de la interfaz.
\newline
\newline
b) El invariante vale en la pre y en la post de todas las funciones de implementacion que se corresponden con funciones de la interfaz (ejemplo: iAgregar).
\newline
\newline
c) El invariante vale en la pre y en la post de todas las funciones auxiliares de la implementacion.
\newline
\newline
d) La funcion de abstraccion provee una explicacion de como se usa la estructura de representacion.
\newline
\newline
e) Por ende, determina la complejidad de las operaciones.
\newline
\newline

\maketitle

\begin{document}
