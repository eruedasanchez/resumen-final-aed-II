\documentclass[10pt,a4paper]{article}
\usepackage[paper=a4paper,hmargin=0.5cm,bottom=1.5cm]{geometry}
\usepackage[utf8]{inputenc}
\usepackage[spanish]{babel}
\usepackage{xargs}
\usepackage{xspace}
\usepackage{caratula}
\usepackage{ifthen}
\usepackage{aed2-tad,aed2-symb,aed2-itef,aed2-diseno}
\usepackage{aed2-tad,aed2-symb,aed2-itef}
\usepackage{algorithmicx, algpseudocode, algorithm}
\usepackage{graphicx}


\materia{Algoritmos y Estructuras de Datos II}
\titulo{Examen Final}
\subtitulo{8 de marzo 2023}
\integrante{Ezequiel Rueda Sanchez}{522/16}{ezequiel.ruedasanchez@gmail.com}

\maketitle

\begin{document}

\section{Ejercicio 1}

Determine si las siguientes afirmaciones son verdaderas o falsas. Justifique.
\newline
\newline
a) El invariante vale en la pre pero no en la post de todas las funciones de la interfaz.
\newline
\newline
b) El invariante vale en la pre y en la post de todas las funciones de implementacion que se corresponden con funciones de la interfaz (ejemplo: iAgregar).
\newline
\newline
c) El invariante vale en la pre y en la post de todas las funciones auxiliares de la implementacion.
\newline
\newline
d) La funcion de abstraccion provee una explicacion de como se usa la estructura de representacion.
\newline
\newline
e) Por ende, determina la complejidad de las operaciones.
\newline
\newline
\textbf{Resolucion}
\newline
\newline
a) La afirmacion es \textbf{verdadera} porque el invariante de representacion se tiene que cumplir en la precondicion y en la postcondicion de todas las funciones de la intefaz a excepcion de las funciones auxiliares definidas para el modulo.
\newline
\newline
b) La afirmacion es \textbf{verdadera} por lo mencionado en el item anterior dado que las funciones auxiliares no son funciones de implemetacion que se corresponden con funciones de la interfaz. 
\newline
\newline
c) La afirmacion es \textbf{falsa} porque justamente el invariante de representacion debe valer en la precondicion y postcondicion de las funciones del modulo pero no en las funciones auxiliares del mismo.
\newline
\newline
d) La afirmacion es \textbf{verdadera} porque nos permite vincular las distintas instancias validas de la estructura de representacion con el valor representado del TAD dado que con los observadores basicos podemos identificar de manera univoca a cualquier instancia.
\newline
\newline
e) COMPLETAR!!!
\newpage

\section{Ejercicio 2}

Un viejo estereo de auto funciona de la siguiente manera. Solo capta radio en las bandas AM, FM y SW (onda corta) y ademas de botones para seleccionar la banda tiene botones para subir y bajar la frecuencia de a 10Hz. Cada banda tiene un rango de frecuencias distinto, por lo que al cambiar de banda vuelve a la frecuencia en la que estaba la ultima vez que se selecciono dicha banda. El volumen esta fijo y tiene 5 botones para memorizar estaciones. Estos botones funcionan de la siguiente manera: si se los pulsa durante dos segundos memorizan la estacion actualmente seleccionada. Si se los presiona menos tiempo van a la estacion memorizada. Se propone la siguiente signtura del TAD \tadNombre{Estereo}.
\newline
\newline
\begin{tad}{\tadNombre{Empresa}}
	\medskip	
	\tadObservadores
	\medskip
	\tadOperacion{banda}{estereo}{nat}{}
	\medskip
	\tadOperacion{frecuencia}{estereo}{nat}{}
	\medskip
	
	\tadGeneradores
	\medskip
	\tadOperacion{nuevo}{}{estereo}{}
	\medskip
	\tadOperacion{cambiar\_frecuencia}{estereo, int}{estereo}{}
	\medskip
	\tadOperacion{cambiar\_banda}{estereo, nat}{estereo}{}
	\medskip
	\tadOperacion{presionar\_boton\_dos\_segundos}{estereo, nat}{estereo}{}
	\medskip
	\tadOperacion{presionar\_boton}{estereo, nat}{estereo}{}
	\medskip
	\medskip
	
	\tadAxiomas[$\forall$]
	
	\tadAlinearAxiomas{suma edades(llega\_empleado(crear\_empleado(d,e,l),a,f)}
	\medskip
	\tadAxioma{suma\_edades(crear(c))}{0}
	\medskip
	\tadAxioma{suma edades(llega\_empleado(crear\_empleado(d,e,l),a,f)}{e + suma\_edades(f)}
	\medskip
	\medskip

\end{tad}

Corregir los errores si los hubiera y presentar la axiomatizacion del observador \TipoVariable{frecuencia}.


\newpage



\end{document}
