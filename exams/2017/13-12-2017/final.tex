\documentclass[10pt,a4paper]{article}
\usepackage[paper=a4paper,hmargin=0.5cm,bottom=1.5cm]{geometry}
\usepackage[utf8]{inputenc}
\usepackage[spanish]{babel}
\usepackage{xargs}
\usepackage{xspace}
\usepackage{caratula}
\usepackage{ifthen}
\usepackage{aed2-tad,aed2-symb,aed2-itef,aed2-diseno}
\usepackage{aed2-tad,aed2-symb,aed2-itef}
\usepackage{algorithmicx, algpseudocode, algorithm}
\usepackage{graphicx}

\materia{Algoritmos y Estructuras de Datos II}
\titulo{Examen Final}
\subtitulo{13 de diciembre 2017}
\integrante{Ezequiel Rueda Sanchez}{522/16}{ezequiel.ruedasanchez@gmail.com}


\begin{document}


\maketitle

\section{Ejercicio 1}

Demostrar por qué la altura de un árbol de Fibonacci es logarítmica respecto a la cantidad de nodos.
\newline
\newline
\textbf{Resolucion}
\newline
\newline
Para demostrar que la altura $h$ de un arbol de Fibonacci es logaritmica respecto a la cantidad de nodos, nos alcanza con probar que este arbol de altura $h$ de menor cantidad de nodos, tienen una cantidad exponencial de nodos en funcion de su altura. 
\newline
\newline
Para esto, podemos probar que la cantidad de nodos de un arbol de Fibonacci $Fibo_{h}$ $=$ $Fibo_{h-1}$ $+$ $Fibo_{h-2}$ $+$ $1$ y que 
\newline
$Fibo_{h}$ $=$ $Fibo_{h-2}$ $-$ $1$ siendo $Fibo_{i}$ es $i$-esimo termino de la sucesion de Fibonacci. Para ver esto, solo hace falta saber que se puede construir un arbol de Fibonacci de altura $h+2$ a partir de unir dos arboles: uno de altura $i$ con otro de altura $i+1$. A partir de esto, probamos que un arbol de $n$ nodos tiene una altura menor a 1.44 $log(n+2)$ $-$ $0.328$ por lo que queda demostrado que tiene una altura de $log(n)$.     
\newline
\newline
En el siguiente grafico, podemos ver como se van conformando los arboles de Fibonacci:

\begin{figure}[h]
\centering
\includegraphics[width=0.6\textwidth]{avl2}%
\caption{Conformacion de los Arboles de Fibonacci}
\label{drivers1}
\end{figure}

Ahora, vemos como se relacionan los Arboles de Fibonacci:

\begin{figure}[h]
	\centering
	\includegraphics[width=0.7\textwidth]{avl3}%
	\caption{Relacion de los Arboles de Fibonacci}
	\label{drivers1}
\end{figure}
\newpage

\section{Ejercicio 2}

Comparar los algoritmos de ordenamiento que son $O(n~log(n))$ en aspectos como uso de memoria adicional, hipótesis sobre el input, etc.
\newline
\newline
\textbf{Resolucion}
\newline
\newline
Los algoritmos de ordenamiento vistos que tienen complejidad $O(n~log(n))$ son \textbf{HeapSort} y \textbf{MergeSort}. Entonces, analizamos cada uno de ellos.
\newline
\newline
El algoritmo de \textbf{HeapSort} no utiliza memoria adicional y con respecto al input, no es estable ya que es ineficaz con respecto a inputs del mismo valor.  
\newline
\newline
El algoritmo de \textbf{MergeSort} utiliza memoria adicional para ir almacenando los arreglos parcialmente ordenados y con respecto al input, generalmente es un algoritmo estable pero puede haber implementaciones que lo hagan inestable en el caso que ante igualdad de valores, se pase al arreglo ordenado el que se encuentra en la segunda mitad.  



\newpage










