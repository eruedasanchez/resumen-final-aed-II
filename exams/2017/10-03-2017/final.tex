\documentclass[10pt,a4paper]{article}
\usepackage[paper=a4paper,hmargin=0.5cm,bottom=1.5cm]{geometry}
\usepackage[utf8]{inputenc}
\usepackage[spanish]{babel}
\usepackage{xargs}
\usepackage{xspace}
\usepackage{caratula}
\usepackage{ifthen}
\usepackage{aed2-tad,aed2-symb,aed2-itef,aed2-diseno}
\usepackage{aed2-tad,aed2-symb,aed2-itef}
\usepackage{algorithmicx, algpseudocode, algorithm}
\usepackage{graphicx}

\materia{Algoritmos y Estructuras de Datos II}
\titulo{Examen Final}
\subtitulo{10 de marzo 2017}
\integrante{Ezequiel Rueda Sanchez}{522/16}{ezequiel.ruedasanchez@gmail.com}


\begin{document}


\maketitle

\section{Ejercicio 1}

Responder Verdadero o Falso. Justificar.
\newline
\newline
a) Lo que hace que una operación sea un observador básico es que deba escribirse en base a los generadores.
\newline
\newline
b) Si una operación rompe la congruencia debe ser transformada en observador básico.
\newline
\newline
c) Dos instancias del mismo TAD pueden ser observacionalmente iguales y aun así ser distinguibles por una operación.
\newline
\newline
d) Si un enunciado dice $"$Siempre que vale $A$ sucede inmediatamente $B$ y $B$ no puede suceder de ninguna otra manera$"$ y la correspondiente axiomatización incluye las operaciones $A$ y $B$ entonces el TAD está mal escrito.
\newline
\newline
