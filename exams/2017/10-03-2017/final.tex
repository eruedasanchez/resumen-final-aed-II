\documentclass[10pt,a4paper]{article}
\usepackage[paper=a4paper,hmargin=0.5cm,bottom=1.5cm]{geometry}
\usepackage[utf8]{inputenc}
\usepackage[spanish]{babel}
\usepackage{xargs}
\usepackage{xspace}
\usepackage{caratula}
\usepackage{ifthen}
\usepackage{aed2-tad,aed2-symb,aed2-itef,aed2-diseno}
\usepackage{aed2-tad,aed2-symb,aed2-itef}
\usepackage{algorithmicx, algpseudocode, algorithm}
\usepackage{graphicx}

\materia{Algoritmos y Estructuras de Datos II}
\titulo{Examen Final}
\subtitulo{10 de marzo 2017}
\integrante{Ezequiel Rueda Sanchez}{522/16}{ezequiel.ruedasanchez@gmail.com}


\begin{document}


\maketitle

\section{Ejercicio 1}

Responder Verdadero o Falso. Justificar.
\newline
\newline
a) Lo que hace que una operación sea un observador básico es que deba escribirse en base a los generadores.
\newline
\newline
b) Si una operación rompe la congruencia debe ser transformada en observador básico.
\newline
\newline
c) Dos instancias del mismo TAD pueden ser observacionalmente iguales y aun así ser distinguibles por una operación.
\newline
\newline
d) Si un enunciado dice $"$Siempre que vale $A$ sucede inmediatamente $B$ y $B$ no puede suceder de ninguna otra manera$"$ y la correspondiente axiomatización incluye las operaciones $A$ y $B$ entonces el TAD está mal escrito.
\newline
\newline
\textbf{Resolucion}
\newline
\newline
a) La afirmacion es \textbf{falsa} porque lo que hace que una operacion sea un observador basico es que permita diferenciar instancias del TAD en clases de equivalencia. Luego, los observadores basicos se axiomatizan en base a sus generadores generalmente. 
\newline
\newline
b) La afirmacion es \textbf{falsa} porque transformarla en observador basico es solo una opcion. Tambien se podria quitarla de las operaciones y asi, evitamos cualquier tipo de inconsistencia.   
\newline
\newline
c) La afirmacion es \textbf{verdadera} y lo justifico con el siguiente ejemplo.
\newline
\begin{tad}{\tadNombre{Negocio}}
	\medskip
	\tadObservadores
	\medskip
	\tadOperacion{totalVendido}{negocio}{nat}{}
	\medskip
	\tadGeneradores
	\medskip
	\tadOperacion{inaugurar}{}{negocio}{}
	\medskip
	\tadOperacion{vender}{negocio, producto}{negocio}{}
	\medskip
	\medskip
	
	\tadOtrasOperaciones
	\medskip
	\tadOperacion{ventas}{negocio}{secu(producto)}{}
	\medskip
\end{tad}
\medskip
\medskip
En este caso, dos instancias del TAD \tadNombre{Negocio} son observacionalmente iguales pero la operacion \TipoVariable{ventas()} permite diferenciar instancias de acuerdo a los axiomas ya que esta funcion rompe la congruencia, es decir, diferencia elementos que quedan en la misma clase de equivalencia de acuerdo con los observadores basicos. Por ejemplo, dos instancias pueden tener el mismo total vendido pero los productos vendidos pueden ser totalmente distintos.   
\newline
\newline
d) La afirmacion es \textbf{...} porque 
\newpage

\section{Ejercicio 2}

En cada uno de los siguientes escenarios, indique qué método de ordenamiento de los estudiados en clase utilizaría y porque.
\newline
\newline
a) Se tiene un arreglo de naturales $A[1,...,n]$ y se desea ordenarlos (de mayor a menor) solamente sí la suma de los $k$ elementos más grandes es menor que $X$. Se desea realizar ésta tarea de forma eficiente, dandola por terminada lo antes posible si la condición no se cumple. (La verificación forma parte de la tarea, no se debe verificar la condición antes de empezar a ordenar)
\newline
\newline
b) Se tiene un arreglo redimensionable de naturales $A[1,...,n]$ y se desea ordenarlos. Sin embargo, durante el proceso de ordenamiento es posible que se agreguen nuevos elementos al final del arreglo.
\newline
\newline
\textbf{Resolucion}
\newline
\newline
a) En este caso, voy a elegir el metodo de \textbf{Heapsort} pero con algunas modificaciones. 

\begin{algorithm}[H]{\textbf{ordenarMayorMenor}(\Inout{A}{arreglo(nat)})}
	\begin{algorithmic}[1]
		\State $suma$ $\gets$ 0             				     \Comment $O(1)$
		\State maxHeap $H$ $\gets$ array2Heap($A$)               \Comment $O(n)$
		\For{$i=0$; $i < tam(A)$; $i++$} 	                     \Comment $O(n)$
		\State $max$ $\gets$ proximo($H$)                        \Comment $O(1)$
		\If{$i == k$ \&\& $!$($suma$ $<$ $X$)}
		\State $i$ $\gets$ $tam(A) - 1$             			 \Comment $O(1)$
		\Else
		\State desencolar($H$)                              \Comment $O(log(n))$
		\State $suma$ $\gets$ $suma$ $+$ $max$                   \Comment $O(1)$
		\State $A[i]$ $\gets$ $max$                              \Comment $O(1)$
		\EndIf
		\EndFor
		
		\medskip
		\Statex \underline{Complejidad:} $O(n ~.~ log(n))$
	\end{algorithmic}
\end{algorithm}

Elegi optar por el metodo de \textbf{Heapsort} dado que la operacion \TipoVariable{arrayToHeap()} permite convertir un arreglo a maxHeap con costo $O(n)$ donde $n$ es el tamaño del arreglo. De este modo, puedo acceder al elemento mas grande del arreglo con costo constante $O(1)$. 
\newline
\newline
Luego, voy a iterar en el peor caso, el ciclo principal for $n$ veces (costo $O(n)$). En cada iteracion, voy a consultar si estoy parado en el $k$ elemento mas gracias a la estructura de maxHeap. Esta condicion tiene costo $O(1)$. En el caso de estar iterando el $k$ elemento mayor del arreglo, consulto si cumple la condicion \TipoVariable{$suma$ $<$ $X$} dado que si esto no se cumple, $i$ toma el valor de $n-1$ para luego aumentar una unidad, pasar a valer $n$, salir del ciclo principal y terminar la ejecucion como pide el enunciado. Caso contrario, desencolo con costo $O(log(n))$ el maximo del $heap$ y lo coloco en su posicion correspondiente en el arreglo (costo $O(1)$) quedando eventualmente el arreglo ordenado de mayor a menor.
\newline
\newline
En el peor caso, voy a desencolar a los $n$ elementos del heap y los voy a colocar en el arreglo por lo que el costo del ciclo principal es $O(n ~.~ log(n))$. 
\newline
\newline
Finalmente, el costo del algoritmo es $O(n)$ $+$ $O(n ~.~ log(n))$ $=$ $O(max\{n,n ~.~ log(n)\})$ $=$ $O(n ~.~ log(n))$ siendo la complejidad asitotica maxima que se puede lograr. 
\newpage
