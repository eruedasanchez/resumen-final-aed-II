\documentclass[10pt,a4paper]{article}
\usepackage[paper=a4paper,hmargin=0.5cm,bottom=1.5cm]{geometry}
\usepackage[utf8]{inputenc}
\usepackage[spanish]{babel}
\usepackage{xargs}
\usepackage{xspace}
\usepackage{caratula}
\usepackage{ifthen}
\usepackage{aed2-tad,aed2-symb,aed2-itef,aed2-diseno}
\usepackage{aed2-tad,aed2-symb,aed2-itef}
\usepackage{algorithmicx, algpseudocode, algorithm}
\usepackage{graphicx}

\materia{Algoritmos y Estructuras de Datos II}
\titulo{Examen Final}
\subtitulo{20 de septiembre 2017}
\integrante{Ezequiel Rueda Sanchez}{522/16}{ezequiel.ruedasanchez@gmail.com}


\begin{document}


\maketitle

\section{Ejercicio 1}

Realice un análisis detallado de los casos del teorema maestro y justifique la conclusión de estos en relación a la condición que satisface.
\newline
\newline
\textbf{Resolucion}
\newline
\newline
Comenzamos mostrando el resultado de las recurrencias en el \textbf{Teorema maestro}.
\newline
\newline
$T(n) = \left\{ \begin{array}{lcc}
	aT(\displaystyle \frac{n}{c}) + f(n) &   si  & n > 1 \\
	\\          1 &   si  & n = 1     \\
\end{array}
\right.$
\newline
\newline
\newline
La solucion es:
\newline
\newline
1) $T(n)$ $=$ $\Theta(n^{log_{c}(a)})$ si $f(n)$ $=$ $O(n^{log_{c}(a) - \epsilon})$ para $\epsilon > 0$
\newline
\newline
2) $T(n)$ $=$ $\Theta(n^{log_{c}(a)}log(n))$ si $f(n)$ $=$ $\Theta(n^{log_{c}(a)})$ para $\epsilon > 0$
\newline
\newline
3) $T(n)$ $=$ $\Theta(f(n))$ si $f(n)$ $=$ $\Omega(n^{log_{c}(a) + \epsilon})$ para $\epsilon > 0$ y $af(\displaystyle \frac{n}{c}) < kf(n)$ para $k < 1$ y $n$ suficientemente grande. 
\newline
\newline
En los casos 1 y 3 es $T(n)$ $\in$ $\Theta(f(n) + n^{log_{c}(a)})$
\newline
\newline
En estos casos:
\newline
\newline
\textbf{.} $a$ es la cantidad de subproblemas a resolver.
\newline
\newline
\textbf{.} $c$ es la cantidad de particiones y $\displaystyle \frac{n}{c}$ es el tamaño de los subproblemas a resolver. 
\newline
\newline
\textbf{.} $f(n)$ es el costo de lo que se hace en cada llamado ademas de los llamados recursivos, asi incluye el costo de la division ($D(n)$) y la combinacion de los resultados ($C(n)$). 
\newline
\newline
\textbf{.} Asumimos que el costo de resolver un caso base cuesta $\Theta(1)$.
\newpage

\section{Ejercicio 2}

Explique la forma general de un algoritmo de Divide \& Conquer. ¿Cualquier algoritmo recursivo es un algoritmo de Divide \& Conquer? ¿Qué condiciones se espera que se satisfagan para que se justifique la aplicación de la técnica algorítmica? Ejemplifique un caso.
\newpage
